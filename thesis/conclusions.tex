\documentclass[sigproc-sp.tex]{subfiles} 
\begin{document}
\section{Conclusioni}
Durante la progettazione del lavoro sono sorti alcuni problemi, dettati soprattutto da necessità pratiche: (1) nonostante la quantità di informazione reperibile sul web sia immensa, il fatto di lavorare per un target che è di lingua italiana restringe enormemente il campo, tanto che molte risorse nei dataset non sono state utilizzate; ciò costringe ad impiegarsi in maniera estremamente maggiore nell’acquisizione di molti più dataset, finendo ovviamente per complicare tutto il sistema; (2) Anche le informazioni disponibili in forma strutturata non sono esenti da difetti: DBpedia, nonostante sia una risorsa utilissima, soffre di un problema che ha Wikipedia in primo luogo. La questione proviene dalle libertà che gli editori hanno nella scrittura dei template per le infobox, e risulta quindi in attributi che, pur essendo semanticamente equivalenti, sono proposti con nomi diversi e possono quindi sfuggire alla progettazione di un sistema di raccolta dati automatico; (3) SPARQL ha il pregio di essere un linguaggio di interrogazione estremamente potente, ma questo consegue naturalmente in una difficoltà nel suo apprendimento da parte dell'essere umano; siamo sicuri che nel prossimo futuro, migliorando le query, riusciremo a ottenere più dati, e soprattutto dati migliori. 

Possiamo inoltre lavorare per migliorare la qualità del contenuto della nostra Knowledge Base in due modi: (1) Seguendo l’esempio di TimeTrails, numerosi dati spazio-temporali potrebbero essere ricavati dal testo semplice e dagli attributi scritti in forma estesa e normalizzati prima del salvataggio nella Knowledge Base, utilizzando ad esempio il tagger temporale HeidelTime\footnote{\url{https://code.google.com/p/heideltime/}} e il Geotagger di MetaCarta\footnote{\url{http://www.metacarta.com/products-platform-geotag.htm}}; (2) considerando con grande importanza l'intervento manuale: perchè l’esperienza offerta all’utente sia migliore possibile, vorremmo infatti predisporre una piattaforma pensata per un gruppo di possibili “esperti”, da utilizzare per effettuare pulizia e arricchimento dei dati, affiancandola magari a CrowdMemories.

Merita una riflessione a parte il discorso sulle categorie di risorse per stimolare la reminiscenza: cosa può essere aggiunto e cosa migliorato? Come già spiegato, attualmente Reminiscens è progettato per lavorare su eventi, immagini, canzoni e personaggi famosi, ma non è detto che questa sia la scelta giusta, e molti altri possono essere i tipi utili a favorire l'afflusso di ricordi; un esempio, che altri sistemi simili hanno implementato, riguarda delle domande poste direttamente all'utente. \textit{Pensieve: Supporting Everyday Reminiscence} contiene ad esempio un'interessante categorizzazione di memory triggers\cite{peesapati2010pensieve}.
\end{document}