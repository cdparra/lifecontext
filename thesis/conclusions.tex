\documentclass[sigproc-sp.tex]{subfiles} 
\begin{document}
\section{Conclusioni}
Durante la progettazione del lavoro sono sorti alcuni problemi, dettati soprattutto da necessità pratiche: (1) nonostante la quantità di informazione reperibile sul web sia immensa, il fatto di lavorare per un target che è di lingua italiana restringe enormemente il campo, tanto che molte risorse nei dataset non sono state utilizzate; questo costringe a impiegarsi in maniera estremamente maggiore nell’acquisizione di molti più dataset, finendo ovviamente per complicare tutto il sistema. (2) Anche le informazioni disponibili in forma strutturata non sono esenti da difetti: DBpedia, nonostante sia una risorsa utilissima, soffre di un problema che ha Wikipedia in primo luogo. La questione proviene dalle libertà che gli editori hanno nella scrittura dei template per le infobox, e risulta quindi in attributi che, pur essendo semanticamente equivalenti, sono proposti con nomi diversi e possono quindi sfuggire alla progettazione di un sistema di raccolta dati automatico.

\subsection{Lavoro futuro}
La natura di Reminiscens è quella di un sistema vasto e in continua evoluzione; molto è quindi il lavoro ancora da fare, per aggiungere funzionalità e migliorare quelle esistenti. Il primo passo è quello di portare a compimento i client di Reminiscens, permettendo ai primi utenti di avvicinarsi alla piattaforma, aggiungendo anche un modulo di confronto tra timeline allo scopo di offrire una componente social a gruppi di persone che hanno caratteristiche, gusti e esperienze di vita simili; ispirati da altri sistemi che si occupano della reminiscenza, verranno aggiunti anche dei trigger in forma di domande poste all’utente, che andranno a diventare una sorta di cornice alle risorse mostrate. Connesso a questo è l’obiettivo di dare la possibilità agli utilizzatori di personalizzare la loro esperienza, caricando del materiale privato in un loro spazio ed editando i booklet per adattarli a specifiche esigenze; nel caso di utenti anziani, ci aspettiamo che operazioni di questo tipo vengano portate a termine con naturalezza con l’aiuto dei loro cari. 

Andrà raffinata anche l’ossatura di Reminiscens: In figura BO si può vedere la nuova architettura della Knowledge Base, che fa un passo verso una forma che rispetta la struttura del web semantico con riferimenti del tutto consistenti. Sempre riguardo il backend, un altro importante step da compiere è il raffinamento dei moduli ETL: seguendo l’esempio di TimeTrails, numerosi dati spazio-temporali potrebbero essere ricavati dal testo semplice e dagli attributi scritti in forma estesa e normalizzati prima del salvataggio nella Knowledge Base, utilizzando ad esempio il tagger temporale HeidelTime e il Geotagger di MetaCarta. A questi seguiranno ovviamente nuove versioni dell’algoritmo LifeContext, forse dettate da particolari esigenze emerse dalla raccolta di nuovi dati. Proprio riguardo quest’ultima, c’è da discutere sulla rimozione dei duplicati: se più avanti molti dataset verranno aggunti, si profila il rischio di raccogliere delle entità duplicate. Questo problema, senza escludere completamente la possibilità di un intervento manuale, potrebbe essere risolvibile tramite il calcolo della “firma” della risorsa che, considerata la varietà di forme nelle quali si possono trovare i dati sul web, deve essere una pensata come una funzione di informazioni “immutabili”. Come già accennato, l’intervento manuale è da trattare con grande importanza: perchè l’esperienza offerta all’utente sia migliore possibile, vorremmo infatti predisporre una piattaforma pensata per un gruppo di possibili “esperti”, da utilizzare per effettuare pulizia e arricchimento dei dati, da affiancare magari a CrowdMemories.
\end{document}