\documentclass[sigproc-sp.tex]{subfiles} 
\begin{document}
\section{Soluzione}
Partendo dai paragrafi precedenti, riusciamo ora a delineare quelle che sono le possibili componenti software da costruire per i nostri scopi. La prima e indispensabile è un indice di risorse storiche, da consultare per ottenere le entità più pertinenti alla vita di un individuo; proprio per decidere cosa è attinente e cosa no, un algoritmo deve essere pensato e implementato; affinchè il risultato dell’implementazione sia scalabile e flessibile, deve essere scelto un meccanismo di accesso aperto e utilizzabile in più frangenti; ultima, ma non meno importante, la visualizzazione dei dati raccolti e filtrati, che devono essere presentati in una forma user-friendly, con attenzione particolare agli utenti di una certa età, che, rappresentando una fetta consistente dei possibili utilizzatori, richiedono una progettazione più attenta.

Nella realizzazione di un sistema capace di rispondere all’utente con dati relativi al tempo e al luogo in cui egli ha vissuto alcuni momenti della sua vita, la prima difficoltà è riscontrabile nella scelta riguardo quale meccanismo utilizzare per recuperare le entità con cui costruire il contesto; per ottenere questo risultato le possibilità si possono ricondurre a tre approcci principali:

\begin{itemize}
\item Sfruttare i linked data per recuperare a runtime le informazioni necessarie: questo è una scelta ambiziosa, che pone molta fiducia nel fatto di riuscire a scavare nel modo giusto tra i datasets per trovare abbastanza elementi da restituire, elementi che siano anche inerenti alla richiesta fatta.
\item Utilizzare i linked data, collegandoci un altro dataset, formato da informazioni raccolte da noi: il problema che vogliamo risolvere riguarda la storia delle persone, e chi meglio delle altre persone può aiutarci a raggiungere questo obiettivo? per questo un grosso aiuto nella ricerca di immagini, eventi e altro potrebbe venire da un opera di crowdsourcing che, immagazzinata sotto forma di entità con un significato e, collegata ad altre sorgenti di dati, andrebbe a completare i nostri bisogni e forse quelli di altri utilizzatori del semantic web.
\end{itemize}

Quest’ultima è la proposta di soluzione più interessante, che si tuffa nel futuro di una rete nella quale informazioni da sorgenti completamente diverse tra di loro vanno a completarsi e impreziosirsi a vicenda, ma che per la natura sperimentale del nostro lavoro va purtroppo considerata solo un possibile sviluppo futuro .
Per poter essere meno dipendenti dalle specifiche RDF e SPARQL, certamente ancora in evoluzione, l’approccio che abbiamo scelto è di per sé il più semplice: perchè non costruire un archivio storico in forma di database, indicizzarlo in maniera corretta e renderlo disponibile tramite un algoritmo di creazione di un contesto intorno a delle coordinate spazio-temporali? Così nasce Reminiscens, il cui funzionamento e struttura sono spiegati qui sotto.

\subsection{Architettura generale}
Reminiscens è un sistema strutturato di varie parti, che lavorano insieme per fornire numerosi servizi. Le funzionalità di raccolta e memorizzazione dati sono svolte da un database, dei moduli software e un’interfaccia grafica: Il database è la componente fondamentale di Reminiscens, contenente tutto il materiale utilizzato dal sistema per restituire un contesto (di cui si parlerà più avanti) all’utente; perchè la Knowledge Base contenga dei dati, questi devono essere raccolti da qualche parte, nel nostro caso il lavoro è fatto da dei moduli ETL, ognuno dei quali si occupa di interrogare una determinata risorsa sul web (e.g. DBpedia); la stessa funzione è svolta da CrowdMemories, una UI sviluppata da Francesco Maturi che utilizza il crowdsourcing per reperire materiale storico direttamente dagli utenti. L’accesso ai dati invece, è implementato tramite delle API con architettura REST, che svolgono quindi la funzione di intermediari nella comunicazione tra i client e la Knowledge Base. La parte di visualizzazione consiste attualmente in una semplice interfaccia in forma di libro sfogliabile, a cui nel futuro si affiancheranno altri servizi; questi ulteriori sviluppi verranno trattati a [\ref{subsec:futurework}].

\subsection{Architettura del mio lavoro (tmp)}
Durante il mio lavoro, mi sono trovato a seguire i dati per tutto il percorso, dalla loro raccolta alla loro visualizzazione, passando per tutta la fase di elaborazione. Di seguito ecco le parti che ho sviluppato:

\subsubsection{Moduli ETL}
L’implementazione da noi scelta è quella di un insieme di moduli ETL (Extract, Transform, Load) standalone che raccolgono ed elaborano dati, dipendenti quasi solo da un modulo che si occupa della comunicazione diretta con la Knowledge Base. La loro importanza è vitale perchè, oltre all’ovvia funzione di recuperare gli elementi che andranno poi restituiti all’utente in forma di contesto, il hanno anche il compito di ricostruire dove possibile le coordinate spazio-temporali dei dati estratti (e.g. tramite il Geocoding di una locazione scritta in maniera testuale). Essendo la lista delle risorse consultabili sul web sempre in crescendo, questa scelta è stata fatta per rendere semplice l’aggiunta e la sostituzione dei moduli, che attualmente supportano la consultazione di Flickr, DBpedia, Catinabib (catalogo di cartoline e incisioni a tema regionale di proprietà della biblioteca comunale di Trento) e di dataset del progetto OpenData Trentino. Proprio grazie a quest’ultima risorsa, si spera che con il passare del tempo, grazie all’aggiunta di nuovi datasets, le informazioni contenute nella Knowledge Base e di conseguenza fornite all’utente possano crescere di quantità e di qualità. 
Indice: Per permettere una lettura veloce dei dati raccolti, il risultato delle estrazioni dal web viene periodicamente indicizzato, così che non sia necessario leggere tutto il db per ottenere un risultato da spedire ai client visuali.

\subsubsection{Api REST}
Per permettere la consultazione della KB senza dover interrogarla direttamente, ho sviluppato delle semplici api che, tramite messaggi HTTP, permettono di fare ricerche a partire da parametri spazio-temporali. Queste si dividono in due tipi:
\begin{itemize}
\item Nel caso sia necessario semplicemente leggere una lista di elementi dal KB, un insieme di chiamate si assicura che al richiedente torni la giusta lista di tutte le immagini o gli eventi o altro combacianti i parametri di input.
\item Se invece si tratta della comunicazione con i client, altre chiamate vanno a leggere l’indice per restituire un contesto. Questo argomento è descritto in maniera specifica a [\ref{subsubsec:lifecontext}].
\end{itemize}

Per questioni più tecniche, a [NONLOSO] è disponibile la documentazione delle api.

\subsubsection{Simple CRUD}
Affiancato ai moduli ETL e alle api è presente un piccolo e semplice sistema con interfaccia minima che effettua CRUD, per aggiungere e modificare manualmente il materiale del db.

\subsubsection{Booklet}
\label{subsubsec:booklet}
Un’ultima parte riguarda il lavoro per comporre un’interfaccia che potesse essere utile a visualizzare i dati restituiti dalle API, indipendente dai client di Reminiscens: chiedendo all’utente una decade e un luogo, quello che viene mostrato è un libricino sfogliabile tramite browser, e contenente un contesto completo in una forma che possa essere familiare a chi lo guarda, soprattutto nel caso dei meno giovani, che mancano di dimestichezza con i mezzi più tecnologici. Il libro restituito è però un insieme di dati impersonali, in quanto per avere un risultato personalizzato la scelta giusta è il vero e proprio client di Reminiscens.

\subsubsection{Algoritmo LifeContext}
\label{subsubsec:lifecontext}
Come definito sopra, il contesto è un insieme di immagini, eventi, personaggi e musica che hanno a che fare con il tempo e il luogo di cui l’utente potrebbe voler scoprire qualcosa, e che potrebbe aiutarlo a raccontare della sua vita. La nostra scelta è stata quella di comporre questo contesto utilizzando cinque entità per tipo, selezionate secondo alcuni criteri all’interno dell’indice.
Partendo dalla knowledge base, la Lifeincontext API crea un contesto intorno ad una lista di eventi. L’elaborazione si divide due casi principali, all’interno dei quali si snodano diversi sottocasi:

\begin{itemize}
\item se le coppie lat-lon possono essere racchiuse all’interno di un cerchio virtuale di raggio 50 km:
\begin{itemize}
\item se la maggior parte delle decadi è uguale, viene creato un unico contesto a partire da una delle coppie lat-lon e da quella decade. questo si può ottenere con una singola query che, ordinando i risultati per pertinenza, seleziona direttamente i primi cinque;
\item se il numero degli eventi di input è minore di tre, viene creato un contesto componendo dei “sotto-contesti”, uno per ogni coppia formata dalle decadi di input e da una delle coppie lat-lon;
\item l’ultimo caso è un’estensione del secondo, infatti, a partire da una lunga lista di coppie, viene creato un contesto componendo tre sotto-contesti tra quelli possibili; questo viene fatto per non inquinare troppo il risultato, che senza il filtraggio appena indicato sarebbe una semplice accozzaglia di elementi che in ultima analisi sarebbero poco correlati tra di loro;
\end{itemize}
\item se le coppie non soddisfano la condizione, vengono riproposti il secondo e il terzo sotto-caso indicati sopra, con la differenza che ogni contesto viene computato a partire dalle coppie così come indicate in input, senza elaborazioni addizionali.
\end{itemize}

Oltre al calcolo a partire dalle sole indicazioni spazio-temporali, un ulteriore versione dell’algoritmo è stata pensata per cercare, filtrare e personalizzare i risultati ottenuti anche in base a delle parole chiave, siano esse argomenti, generi musicali, interessi dell’utente e oggetto della risorsa: ad esempio, una richiesta riguardante Trento nel decennio 1980, con keyword “calcio”, dovrebbe restituire tra le altre una voce sul campionato di calcio serie C2 1984-85, quando la squadra locale venne promossa in Serie C1 dopo uno spareggio vinto contro l'Ospitaletto.
Grazie a queste poche informazioni, fornite al sistema dall’utente in fase di configurazione del profilo, l’API dovrebbe consegnare del materiale interessante ed evocativo di momenti del suo passato, stimolando così il racconto di episodi della sua vita.

\subsection{Framework e librerie}
Per realizzare queste componenti, mi sono avvalso di diverse tecnologie, elencate qui sotto per sezione.
Moduli ETL e Indice: Tutti i moduli sono realizzati in Java utilizzando Apache HttpComponents per gli scambi di messaggi HTTP, Hibernate, Apache Jena per poter interrogare tramite SPARQL, le API di Google Maps per risolvere tutti i problemi relativi alla risoluzione sia di coordinate in luoghi che il contrario, Google Gson per il parsing di messaggi JSON e le API di Youtube per associare un video musicale a ogni canzone nella KB.
Api REST: la scrittura di queste librerie mi è stata facilitata dall’utilizzo di Ruby unito al framework Sinatra, con l’ausilio di ActiveRecord e del linguaggio di templating RABL. Al fine di eseguire alcuni calcoli complessi, ho eseguito il porting di uno script per la conversione di coordinate geografiche in UTM\footnote{\url{http://home.hiwaay.net/~taylorc/toolbox/geography/geoutm.html}} e ho adattato un implementazione Ruby dell’algoritmo lineare di Megiddo per il calcolo del Minimum Enclosing Circle di un set di punti\footnote{\url{http://www.dseifert.net/code/mec/index.html}}.
CRUD: Questa è un’applicazione Ruby on Rails che si avvale di una semplice UI per la modifica manuale dei dati, ma che permette eseguire le stesse operazioni programmaticamente, inviando e ricevendo JSON.
Booklet: L’interfaccia del booklet si basa su “20 things I learned about browsers and the web”\footnote{\url{http://www.20thingsilearned.com/}}, applicazione web HTML5 resa open source e riadattata per le nostre necessità.

\subsection{Stato della Knowledge Base}
Per questa prima parte di sviluppo, anche pensando a possibili fasi di testing, abbiamo voluto concentrarci sulla raccolta di materiale relativo alla zona del Trentino, con maggiore attenzione alle zone di Trento, Rovereto e Pergine Valsugana; nella realtà dell’implementazione, questo significa che, dove possibile, i dati sono stati filtrati per luogo, con alcune eccezioni: per quanto riguarda la musica ad esempio, abbiamo fatto la scelta di cercare anche musica straniera, data la facilità con cui un abitante locale può venire a contatto con canzoni che non provengono dall’Italia. Al tempo della scrittura di questo documento, la Knowledge Base di Reminiscens contiene 3244 entità, tra immagini, canzoni, eventi e personaggi famosi. Tabella \ref{tab:entitykind} mostra quante di queste risorse appartengono a ogni tipo, mentre tabella \ref{tab:entitysource} mostra la provenienza delle stesse. Entrambe le tabelle mostrano che, nonostante il lavoro da noi compiuto sia ancora in uno stato di sviluppo, (1) è stato molto più semplice raccogliere elementi di tipo musicale (un piccolo progetto che ha lavorato con la musica come terapia è illustrato brevemente a [\ref{subsec:stimuli}]), (2) Il web semantico e il crowd si dimostrano due fonti di valore inestimabile per gli scopi che ci siamo posti, soprattutto considerando il fatto di averle sfruttate in maniera non intensiva; i dati qui mostrati sono solo uno degli input che è necessario analizzare approfonditamente per scegliere quale strada tracciare per il futuro di Reminiscens.
\end{document}