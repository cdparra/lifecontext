\documentclass[sigproc-sp.tex]{subfiles} 
\begin{document}
\section{Introduzione}
E’ successo a tutti, almeno una volta, di ritrovarsi a raccontare a qualcuno (amico, familiare o altro), episodi della propria vita passata: l’atto di raccogliere dalla memoria esperienze passate per condividerle con qualcuno, rendendo significativo il rapporto tra i due attori della conversazione, è chiamato reminiscenza, ed è riscontrabile in tutte le età dell’essere umano. Coleman (1974) distinse la reminiscenza in varie tipologie a seconda del modo in cui essa viene evocata e del suo scopo: tra i diversi tipi troviamo quello detto “life review”, che riguarda proprio la rievocazione di memorie lontane nel tempo, al fine di ricostruire quella che è stata la propria vita. Questo fenomeno assume rilevanza massima per gli anziani: è infatti studiato come terapia per i malati di Alzheimer o affetti da altre forme di demenza, dato che il ricordare eventi piacevoli del passato aiuta a ristorare l’autostima e la soddisfazione personale, combattendo la malattia. Ma non è solo questo: la reminiscenza, in quanto interazione con un altro soggetto, può essere utile anche per combattere la depressione e l’isolamento in cui gli anziani spesso si trovano, favorendo allo stesso tempo un rapporto faccia a faccia, positivo per entrambi i partecipanti.
Indicata la positività del fenomeno, è doveroso fare i conti con il fatto che nella maggior parte dei casi nasce in maniera spontanea, dando vita a un quesito importante: come trovare gli stimoli giusti per innescare la reminiscenza? Crediamo che la fonte più consona a fornire l’input adatto a farla scattare sia quella in cui lavoriamo più o meno tutti i giorni, cioè il web. Dopo più di vent’anni di internet, la nostra storia collettiva è quasi tutta online: se oltre a questo consideriamo l’espansione dei dataset disponibili in forma strutturata, è chiaro che raccogliere da lì materiale audiovisivo, eventi e personaggi storici può fornire delle importanti fondamenta da cui partire.
Ma possedere questi dati non è abbastanza: perchè gli stimoli siano effettivamente significativi, è necessario ideare un meccanismo atto a identificare quale parte dei contenuti raccolti dal web è rilevante per la storia della vita di una persona. Proprio a partire da questo problema si sviluppa il mio lavoro, esplicitato nei seguenti paragrafi, ma la sua trattazione ha bisogno di una digressione sulle tecnologie che nei diversi ambiti rappresentano il progresso raggiunto.
\vspace*{1\baselineskip}
\end{document}