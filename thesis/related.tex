\documentclass[sigproc-sp.tex]{subfiles} 
\begin{document}
\section{Lavori correlati}
Reminiscens non è di certo il primo lavoro compiuto con l’IT al servizio della reminiscenza; discuteremo brevemente alcuni approcci qui di seguito.

\subsection{Stimolare la reminiscenza}
\label{subsec:stimuli}
Pensieve\footnote{\url{http://pensieve.cornellhci.org/}} è un sistema ideato per inviare a intervalli regolari degli stimoli a ricordare. Tali stimoli possono essere interattivi (come una domanda, e.g. “Ti ricordi quando hai imparato a cucinare?”), oppure semplicemente visuali: in questo ultimo caso, è richiesto un collegamento dell’account Pensieve con importanti e popolari servizi, come Picasa, Flickr e Twitter, dai quali Pensieve provvede a estrarre le immagini e i post che verranno utilizzati come contenuto per inviare i trigger all’utente. Tra i risultati del progetto, è stato notato che le persone preferiscono un tipo di reminiscenza di tipo sociale, ma è conseguibile in maniera migliore tramite un'interazione faccia a faccia; questo e gli altri risultati di Pensieve sono descritti in \cite{cosley2012experiences}.

Simile a questo possiamo trovare anche il social network Proust\footnote{\url{http://www.proust.com/}}, che è basato solo su quesiti riguardanti la vita di un individuo e sulle risposte che lo stesso dà. Le risposte, arricchite con informazioni su luogo e tempo, vengono utilizzate per comporre la storia dell’utente, e visualizzate in forma di libro da mostrare agli altri utenti. Proprio la condivisione di queste storie, unita alla possibilità che determinati utenti hanno di chiedere al diretto interessato qualcosa sul suo passato, costituisce il punto di forza di Proust.

Un progetto molto interessante, che si discosta da quelli appena presentati per il suo non essere un sistema informatico, è Alive Inside\footnote{\url{http://www.ximotionmedia.com/}}, che presenta gli effetti per così dire miracolosi, su alcuni anziani malati di Alzheimer e affetti da demenza senile, della musica da loro ascoltata in gioventù. Il risultato è un documentario d’effetto che mostra come essi vengano “svegliati” dalla musica, che fa riaffiorare in un lampo ricordi del passato e li rende in grado di rispondere a domande, in un modo che fino a pochi istanti prima sembrava impossibile: l’obiettivo del film è infatti sensibilizzare le case di cura e gli ospedali, ma anche le famiglie, all’uso di questa terapia semplice ed economica, quanto efficace.

\subsection{Raccolta dei dati}
Numerosi sono stati i tentativi di costruire una banca dati culturale, e un esempio è Memoro\footnote{\url{http://www.memoro.org/}}, piattaforma tutta italiana ma ormai estesa e localizzata in molte regioni del mondo che, prendendo spunto dalla pratica del racconto delle proprie esperienze, tipica di genitori e nonni nei confronti di figli e nipoti, si propone come un contenitore di clip audio e video raccolti dagli utenti, clip che contengono la memoria di vite vissute secondo usanze e valori di un'altra epoca, rendendola disponibile a chiunque abbia 10 minuti di tempo da spendere.

Una menzione speciale va a Live Memories\footnote{\url{http://www.livememories.org/}}, progetto locale coordinato da FBK, Università di Trento e Università di Southampton e orientato alla costruzione di un archivio multimediale eterogeneo, raccogliendo dati da sorgenti molto diverse tra loro. Il progetto, che ha avuto luogo dal 2008 al 2011, si è concluso dopo aver coinvolto, oltre agli enti accademici, anche il quotidiano l’Adige, la rivista Vita Trentina e il consiglio comunale di Trento.

\subsection{Visualizzazione dei dati}
I due modi più naturali di rappresentare entità caratterizzate da coordinate spazio-temporali sono (1) tramite punti su una mappa e (2) con l’utilizzo di timeline; il secondo è l’approccio scelto da Project Greenwich\footnote{\url{http://projectgreenwich.research.microsoft.com/}}, realizzato da Microsoft Research Cambridge. Tramite il login a Facebook, permette di creare delle timeline con le proprie fotografie, arricchirle collegandoci delle pagine da Wikipedia, confrontarle con le timeline create da altri utenti e condividerle con famiglia e amici. Il sistema è stato studiato per permettere agli utenti di riflettere sul loro passato, magari guadagnando consapevolezza, fornendo anche gli strumenti per organizzare i contenuti in maniera creativa ed espressiva.
\end{document}