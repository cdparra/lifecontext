\section{Introduzione}
\`{E} successo a tutti, almeno una volta, di ritrovarsi a raccontare a qualcuno (amico, familiare o altro), episodi della propria vita passata: l’atto di raccogliere dalla memoria esperienze passate per condividerle con qualcuno, rendendo significativo il rapporto trai due attori della conversazione, \`{e} chiamato reminiscenza. Se questo fenomeno è importante durante buona parte della vita, assume rilevanza massima per gli anziani; \`{e} infatti studiato anche come aiuto per i malati di Alzheimer, in quanto il ricordare diventa come un’isola nella quale la persona può trovare degli appigli. Ma non \`{e} solo questo: la reminiscenza, in quanto interazione con un altro soggetto, può essere utile anche per combattere l’isolamento in cui gli anziani spesso si trovano, favorendo allo stesso tempo un rapporto faccia a faccia, positivo per entrambi gli attori.
Indicata la positività del fenomeno, \`{e} doveroso fare i conti con il fatto che nella maggior parte dei casi nasce in maniera spontanea, dando vita a un quesito importante: come trovare gli stimoli giusti per innescare la reminiscenza? Crediamo che la fonte più adatta a fornire l’input adatto a farla scattare sia quello in cui lavoriamo più o meno tutti i giorni, cio\`{e} il web. Con l’enorme quantità di informazioni e archivi storici presenti, considerando l’espansione dei dataset disponibili in forma strutturata, raccogliere materiale audiovisivo, eventi e personaggi storici può fornire delle importanti fondamenta da cui partire.
Ma possedere questi dati non \`{e} abbastanza: perch\'{e} gli stimoli siano effettivamente significativi, \`{e} necessario ideare un meccanismo atto a identificare quale parte dei contenuti raccolti dal web \`{e} rilevante per la storia della vita di una persona. Proprio a partire da questo problema si sviluppa il mio lavoro, esplicitato nei seguenti paragrafi, ma la sua trattazione ha bisogno di una digressione sulle tecnologie che nei diversi ambiti rappresentano il progresso raggiunto.