\begin{abstract}
Nella vita collezioniamo moltissimi ricordi, innumerevoli esperienze che plasmano le persone che siamo e che saremo. Con il passare degli anni aumenta per ognuno l'importanza della rievocazione di queste memorie, che possono avere una doppia valenza, in quanto terapeutiche per combattere la senilità e utili nella trasmissione ai posteri di quel che era. Proprio l'utilità di questo fenomeno spontaneo, chiamato reminiscenza, fa sorgere una domanda: come è possibile far sorgere la reminiscenza in un individuo? Questo documento contiene le scelte di implementazione che hanno portato alla costruzione di Reminiscens, una piattaforma sociale (ancora in fase di sviluppo) per favorire l'afflusso di ricordi provenienti dalla vita di una persona.
OCCHIO - LIFE REVIEW
\end{abstract}